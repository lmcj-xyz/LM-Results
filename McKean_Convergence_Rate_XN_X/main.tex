\documentclass{article}
\usepackage{amsmath}
\usepackage{amsfonts}
\usepackage{amsthm}
\usepackage[utf8]{inputenc}
\usepackage[backend=biber, style=numeric, url=false, isbn=false, doi=false]{biblatex}
\usepackage{bbm}
\usepackage[margin=4cm]{geometry}
\usepackage{hyperref}
\usepackage[capitalise]{cleveref}
\usepackage{todonotes}

\newtheorem{proposition}{Proposition}
\newtheorem{lemma}{Lemma}
\newtheorem{definition}{Definition}

\allowdisplaybreaks

\title{Convergence rate of $X^N - X$ for McKean equations}
\author{Luis Mario Chaparro Jaquez}

\addbibresource{../references.bib}

\begin{document}

\maketitle

\listoftodos

\section{What is this?}

An adaptation of
\cite[Proposition 3.1]{de_angelis_numerical_2020}
for the case of McKean SDEs
proposed in
\cite{issoglio_mckean_2021}.

The proof builds on a number of results presented in the sections below.

\section{Some useful definitions and results}

Here we present some results and definitions to refer on the text.

%%%%%%%%%%%%%%%%%%%%%%%%%%%%%%
% Definition Local time
%%%%%%%%%%%%%%%%%%%%%%%%%%%%%%
\begin{definition}
For any real-valued continuous semi-martingale, the local time at zero
$L^{0}_{t} (\bar{Y})$
is defined as
\begin{equation}
    L^0_t (\bar{Y}) = \lim_{\epsilon \to 0} \frac{1}{2 \epsilon} \int_0^t \mathbbm{1}_{\{|\bar{Y}| \leq \epsilon\}} d \langle \bar{Y} \rangle_s
    \text{, }
    \mathbb{P}
    \text{-a.s.}
\end{equation}
For all
$t \geq 0$.
\end{definition}

%%%%%%%%%%%%%%%%%%%%%%%%%%%%%%
% Local time at zero, Lemma 5.1
%%%%%%%%%%%%%%%%%%%%%%%%%%%%%%
The first result,
\cite[Lemma 5.1]{de_angelis_numerical_2020},
is not necessary to prove for this particular setting since the result holds for any semi-martingale, I include it here for self-containment reasons.

\begin{lemma}
\todo{Type down this lemma}
[Lemma 5.1]
For any
$\epsilon \in (0,1)$
and any real-valued, continuous semi-martingale

\end{lemma}

Let us introduce the original and regularised Kolmogorov equations.

%%%%%%%%%%%%%%%%%%%%%%%%%%%%%%
% Kolmogorov equations
%%%%%%%%%%%%%%%%%%%%%%%%%%%%%%
\begin{definition}
	[Kolmogorov equations]
For
$\beta \in (0, 1/2)$
let
$b \in C_T \mathcal{C}^{-\beta}$,
$u, u^N \in C_T\mathcal{C}^{(1 + \beta)+}$, 
and
$b^N \to b$
as
$N \to \infty$
in
$C_T \mathcal{C}^{-\beta}$
	The equations

	\begin{equation}
	\begin{cases}
	\partial u_i + \frac{1}{2} b_i \Delta u_i = \lambda u_i - b_i
	\\
	u_i(T) = 0,
	\end{cases}
	\end{equation}
	
	\begin{equation}
	\begin{cases}
	\partial u^N_i + \frac{1}{2} b^N_i \Delta u^N_i = \lambda u^N_i - b^N_i
	\\
	u^N_i(T) = 0.
	\end{cases}
	\end{equation}

	are called Kolmogorov and regularised Kolmogorov equations.
	Here written component wise.
	\label{def:kolmogorov_eqns}
\end{definition}

%%%%%%%%%%%%%%%%%%%%%%%%%%%%%%
% Lemma 5.2
%%%%%%%%%%%%%%%%%%%%%%%%%%%%%%
\section{Lemma 5.2}

Adaptation of Lemma 5.2 builds on the following result:

%%%%%%%%%%%%%%%%%%%%%%%%%%%%%%
% Bound for the rho-equivalent norm of the difference u-uN
%%%%%%%%%%%%%%%%%%%%%%%%%%%%%%
\begin{proposition}
Let
$u, u^N$
be (mild) solutions to the Kolmogorov equations from
\cref{def:kolmogorov_eqns}
then as
$N \to \infty$

\begin{equation}
\| u_i - u^N_i \|^{(\rho)}_{C_T \mathcal{C}^{1 + \alpha}}
	\leq
	\frac{c T^{\frac{1 - \beta - \alpha}{2}} \| b_i - b^N_i \|_{C_T\mathcal{C}^{-\beta}} (\| u_i \|_{C_T\mathcal{C}^{1 + \alpha}} - 1) }{1 - c \rho^{\frac{\alpha + \beta - 1}{2}} (\| b \|_{C_T\mathcal{C}^{-\beta}} + \lambda)}
\end{equation}

for
$\rho \geq \rho_0$,
where

\begin{equation}
	\rho_0 = 2 c ( \|  b_i \|_{C_T \mathcal{1+\alpha}} + \lambda ) ^ { \frac{ 2 }{ \alpha + \beta + 1}}
\end{equation}

and
$\lambda \neq 0$.
\todo{Is it just different to zero or greater?}

\label{prop:diff_u_uN}
\end{proposition}

\begin{proof}

	See that
	$u^N(T) = u(T) = 0$,
	and in
	\cite{issoglio_pde_nodate},
	set
	$g^N, g$
	as
	$b^N, b$
	respectively.
	See that
	$b^N \to b$.
	Then let us reformulate the rest of the aforementioned result for
	$\lambda \neq 0$.

	As
	$u^N, u$
	are mild solutions, we have

	\begin{align*}
		u_i(t) - u^N_i(t)
		&=
		P_{T-t} (u_i(T) - u^N_i(T))
		\\
		&+
		\int_t^T P_{s-t}(\nabla u_i b_i - \nabla u^N_i b^N_i) ds
		\\
		&-
		\int_t^T P_{s-t}(\lambda u_i + b_i - \lambda u^N_i + b^N_i) ds
		\\
		&=
		\int_t^T P_{s-t}(\nabla u_i b_i - \nabla u^N_i b^N_i) ds
		\\
		&-
		\lambda  \int_t^T P_{s-t}(u_i - u^N_i) ds
		\\
		&-
		\int_t^T P_{s-t}(b_i - b^N_i) ds
		\\
		&=
		\int_t^T P_{s-t}[(\nabla u_i b_i - \nabla u_i b^N_i) + (\nabla u_i b^N_i - \nabla u^N_i b^N_i)] ds
		\\
		&-
		\lambda  \int_t^T P_{s-t}(u_i - u^N_i) ds
		\\
		&-
		\int_t^T P_{s-t}(b_i - b^N_i) ds
		\\
		&=
		\int_t^T P_{s-t}(\nabla u_i b_i - \nabla u_i b^N_i)  ds
		\\
		&+
		\int_t^T P_{s-t}(\nabla u_i b^N_i - \nabla u^N_i b^N_i) ds
		\\
		&-
		\lambda  \int_t^T P_{s-t}(u_i - u^N_i) ds
		\\
		&-
		\int_t^T P_{s-t}(b_i - b^N_i) ds
	\end{align*}

	Now let us compute the $\rho$-equivalent norm of
	$u-u^N$,
	for some
	$\alpha > \beta$

	\begin{align*}
		\| u_i - u^N_i \| ^ {(\rho)}_{C_T \mathcal{C}^{-\beta}}
		&=
		\sup_{0\leq t \leq T} e^{-\rho(T-t)} \| u(t) - u^N(t) \|_{1 + \alpha}
		\\
		&\leq
		\sup_{0\leq t\leq T} e ^{-\rho(T-t)} \| \int_t^T P_{s-t}(\nabla u_i b_i - \nabla u_i b^N_i)  ds \|_{1+\alpha}
	     	\\ 
		&+
		\sup_{0\leq t\leq T} e ^{-\rho(T-t)} \| \int_t^T P_{s-t}(\nabla u_i b^N_i - \nabla u^N_i b^N_i) ds\|_{1+\alpha}
	     	\\ 
		&-
		\sup_{0\leq t\leq T} e ^{-\rho(T-t)} \| \lambda  \int_t^T P_{s-t}(u_i - u^N_i) ds\|_{1+\alpha}
	     	\\ 
		&-
		\sup_{0\leq t\leq T} e ^{-\rho(T-t)} \| \int_t^T P_{s-t}(b_i - b^N_i) ds\|_{1+\alpha}
		.
	\end{align*}

	Let us take each term from the right hand side of the inequality and bound them.

	For the first term, using
	$\gamma + 2\theta = 1 + \alpha$,
	$\gamma = -\beta$,
	$\theta = \frac{1 + \alpha + \beta}{2}$,
	$\| P_t f \|_{\gamma + 2 \theta} \leq c t ^{-\theta} \| f \|_{\gamma} \|$
	and
	$\| \nabla g \|_\xi \leq c \|g \|_{\xi + 1}$

	\begin{align*}
		\sup_{0\leq t\leq T} e ^{-\rho(T-t)} \| \int_t^T P_{s-t}(\nabla u_i b_i - \nabla u_i b^N_i)  ds \|_{1+\alpha}
		&\leq
		\sup_{0\leq t\leq T} e ^{-\rho(T-t)} \int_t^T (s - t)^{-\theta} \| \nabla u_i\|_\alpha \| b_i - b^N_i \|_{-\beta}
		\\
		&\leq
		c \| u_i \|_{C_T \mathcal{C}_{1 + \alpha}} \| b_i - b^N_i \|_{C_T \mathcal{C}^{-\beta}} \sup_{0 \leq t \leq T} e^{-\rho(T-t)}(T-t)^{\frac{1 - \beta -\alpha}{2}}
		\\
		&\leq
		c T^{\frac{1 - \beta - \alpha}{2}} \| u_i \|_{C_T \mathcal{C}_{1 + \alpha}} \| b_i - b^N_i \|_{C_T \mathcal{C}^{-\beta}}
	\end{align*}

	For the second term, see that for
	$N \to \infty$,
	we have
	$\| b^N \|_{C_T \mathcal{C}^{-\beta}} \leq 2 \| b \|_{C_T \mathcal{C}^{-\beta}}$

	\begin{align*}
		\sup_{0\leq t\leq T} e ^{-\rho(T-t)} \| \int_t^T P_{s-t} b^N_i(\nabla u_i - \nabla u^N_i ) ds\|_{1+\alpha}
		&\leq
		c \sup_{0\leq t\leq T} \int_t^T(s - t)^{-\theta} e^{-\rho(T-t)} 2 \| b_i \|_{-\beta} \| \nabla u_i - \nabla u^N_i \|_{\alpha} ds
		\\
		&\leq
		c \| b_i \|_{C_T \mathcal{C}^{-\beta}} \| u_i - u^N_i \|_{C_T \mathcal{C}^{-\beta}}^{(\rho)} \int_t^T(s - t)^{-\theta} e^{-\rho(T-t)} ds
		\\
		&\leq
		c \| b_i \|_{C_T \mathcal{C}^{-\beta}} \| u_i - u^N_i \|_{C_T \mathcal{C}^{-\beta}}^{(\rho)} \rho^{\frac{\alpha + \beta - 1}{2}}
	\end{align*}

	For the third term, which is the one that difers from the proof in
	\cite{issoglio_pde_nodate}
	we need to use that
	$\| P_t f \|_{\gamma + 2\theta} \leq ct^{-\theta} \| f \|_{\gamma}$,
	and in this case we have
	$\gamma + 2 \theta = 1 + \alpha$
	and
	$\gamma = 1 + \alpha$,
	so that
	$\theta = 0$
	because
	$u, u^N \in C_T \mathcal{C}^{1+\alpha}$,
	so we will have

	\begin{align*}
		\sup_{0\leq t\leq T} e ^{-\rho(T-t)} \| \lambda  \int_t^T P_{s-t}(u_i - u^N_i) ds\|_{1+\alpha}
	\end{align*}

\end{proof}

%%%%%%%%%%%%%%%%%%%%%%%%%%%%%%
% Adaptation of Lemma 5.2
%%%%%%%%%%%%%%%%%%%%%%%%%%%%%%
\begin{proposition}
[Adaptation of Lemma 5.2]
Let
$\beta \in (0, 1/2)$
and
$b \in C_T \mathcal{C}^{-\beta}$.
Let 
$u, u^N \in C_T\mathcal{C}^{(1 + \beta)+}$
be (mild) solutions to the Kolmogorov equations from
\cref{def:kolmogorov_eqns}

Assume that for some $\alpha > \beta$

\begin{equation}
	\| u - u^N \|^{(\rho)}_{C_{T} \mathcal{C}^{1+\alpha}}
	\leq
	c(\rho)
	\| b - b^N \|_{C_{T} \mathcal{C}^{-\beta}}
	.
\end{equation}

With
$c(\rho)$
as in
\cref{prop:diff_u_uN}
and
$\rho_0$
is large enough such that
$c(\rho) > 0$
for all
$\rho > \rho_0$.

Then for all
$t \in [0, T]$


\begin{enumerate}
	\item
	$
	\| u^N(t) - u(t) \|_{L^\infty}
	\leq
	\kappa_\rho \| b - b^N \|_{C_T\mathcal{C}^{-\beta}}
	$

	\item
	$
	\| \nabla u^N(t) - \nabla u(t) \|_{L^\infty}
	\leq
	\kappa_\rho \| b - b^N \|_{C_T\mathcal{C}^{-\beta}}
	$
\end{enumerate}

with
$\kappa_\rho = c \cdot c(\rho) \cdot e^{\rho T}$.

\label{prop:diff_X_XN}
\end{proposition}

\section{Lemma 5.3}

\section{Proposition 5.4}

\section{Proposition 3.1 (main result)}

\printbibliography

\end{document}
