\documentclass{article}
\usepackage[utf8]{inputenc}
\usepackage[backend=biber, style=numeric, url=false, isbn=false, doi=false]{biblatex}
\usepackage{bbm}

\newtheorem{lemma}{Lemma}
\newtheorem{definition}{Definition}

\title{Convergence rate of $X^N - X$ for McKean equations}
\author{Luis Mario Chaparro Jaquez}

\addbibresource{../references.bib}

\begin{document}

\maketitle

\section{What is this?}

An adaptation of
\cite[Proposition 3.1]{de_angelis_numerical_2020}
for the case of McKean SDEs
proposed in
\cite{issoglio_mckean_2021}.

The proof builds on a number of results presented in the sections below.

\begin{definition}
For any real-valued continuous semi-martingale, the local time at zero
$L^0_t (\bar{Y})$
is defined as
\begin{equation}
    L^0_t (\bar{Y}) = \lim_{\epsilon \to 0} \frac{1}{2 \epsilon} \int_0^t \mathbb{1}_{|\bar{Y}| \leq \epsilon} d \langle \bar{Y} \rangle_s
    \text{, }
    \mathbb{P}
    \text{-a.s.}
\end{equation}
For all
$t \geq 0$.
\end{definition}

\section{Lemma 5.1}
The first result,
\cite[Lemma 5.1]{de_angelis_numerical_2020},
is not necessary to prove for this particular setting since the result holds for any semi-martingale, I include it here for self-containment reasons.

\begin{lemma}
[Lemma 5.1]
For any
$\epsilon \in (0,1)$
and any real-valued, continuous semi-martingale

\end{lemma}

\section{Lemma 5.2}

\section{Lemma 5.3}

\section{Proposition 5.4}

\section{Proposition 3.1 (main result)}

\printbibliography

\end{document}
