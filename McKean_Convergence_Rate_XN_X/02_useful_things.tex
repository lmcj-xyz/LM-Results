\section{Some useful definitions and results}

Here we present some results and definitions to refer on the text.

%%%%%%%%%%%%%%%%%%%%%%%%%%%%%%
% Definition Local time
%%%%%%%%%%%%%%%%%%%%%%%%%%%%%%
\begin{definition}{Local time at zero}{local_time_zero}
  For any real-valued continuous semi-martingale, the local time at zero
  $L^{0}_{t} (\bar{Y})$
  is defined as
  \begin{equation}
    L^0_t (\bar{Y}) = \lim_{\epsilon \to 0} \frac{1}{2 \epsilon} \int_0^t \mathbbm{1}_{\{|\bar{Y}| \leq \epsilon\}} d \langle \bar{Y} \rangle_s
    \text{, }
    \mathbb{P}
    \text{-a.s.}
  \end{equation}
  For all
  $t \geq 0$.
\end{definition}

%%%%%%%%%%%%%%%%%%%%%%%%%%%%%%
% Local time at zero, Lemma 5.1
%%%%%%%%%%%%%%%%%%%%%%%%%%%%%%
The first result,
\cite[Lemma 5.1]{de_angelis_numerical_2020},
is not necessary to prove for this particular setting since the result holds for any semi-martingale, I include it here for self-containment reasons.

\begin{lemma}{Bound for local time at zero for a semi-martingale}{local-time-at-0}
%%%%\todo{Type down this lemma}
For any
$\epsilon \in (0,1)$
and any real-valued, continuous semi-martingale
$Z$
we have

\begin{align*}
    \mathbb{E}[L^0_t(Z_s)] \leq & 4 \epsilon
    - \mathbb{E} \left[ \int_0^t \left( \mathbbm{1}_{\{Z_s \in (0, \epsilon)\}} + \mathbbm{1}_{\{Z_s \geq \epsilon\}} e^{1 - Z_s/\epsilon} \right) dZ_s \right]
    \\
    &+ \frac{1}{\epsilon} \mathbb{E} \left[ \int_0^t \mathbbm{1}_{\{Z > \epsilon} e^{1 - Z_s/\epsilon\}} d \langle Z \rangle_s \right].
\end{align*}

\end{lemma}

Let us introduce the original and regularised Kolmogorov equations.

%%%%%%%%%%%%%%%%%%%%%%%%%%%%%%
% Kolmogorov equations
%%%%%%%%%%%%%%%%%%%%%%%%%%%%%%
\begin{definition}{Kolmogorov equations}{kolmogorov_eqns}

For
$\beta \in (0, 1/2)$
let
$b \in C_T \mathcal{C}^{-\beta}$,
$u, u^N \in C_T\mathcal{C}^{(1 + \beta)+}$, 
and
$b^N \to b$
as
$N \to \infty$
in
$C_T \mathcal{C}^{-\beta}$
	The equations
	%EI: mistake in PDEs, correct please

	\begin{equation}
	\begin{cases}
	\partial u_i + \frac{1}{2} b_i \Delta u_i = \lambda u_i - b_i
	\\
	u_i(T) = 0,
	\end{cases}
	\end{equation}
	
	\begin{equation}
	\begin{cases}
	\partial u^N_i + \frac{1}{2} b^N_i \Delta u^N_i = \lambda u^N_i - b^N_i
	\\
	u^N_i(T) = 0.
	\end{cases}
	\end{equation}

	are called Kolmogorov and regularised Kolmogorov equations.
	Here written component wise.
\end{definition}

%%%%%%%%%%%%%%%%%%%%%%%%%%%%%%
% Bounds for the gradient of u
%%%%%%%%%%%%%%%%%%%%%%%%%%%%%%

\begin{lemma}{Bounds for $ \nabla u, \nabla u^N $}{lemma:bounds_gradients}
	Let
	\todo{type this result which is lemma 4.2 in the paper}
\end{lemma}

